Systems driven out of thermodynamic equilibrium are capable of demonstrating extraordinarily varied behaviors.  One such behavior is enhanced discrimination: the ratio of nonequilibrium steady-state occupancies can deviate far from what is allowed by their energetic differences alone.  The cost of this flexibility is parametric complexity.  Nonequilibrium steady-state occupancies will generally depend on the system's full parametric details.

Enhanced discrimination is particularly important in a biological context, where biochemical reactions achieve extraordinary fidelity by discriminating between competing substrates having only marginally different equilibrium energies.  For example, equilibrium energy differences between nucleotides competing in a DNA replication reaction predict replication error rates of $\sim10^{-4}$; but observed rates are $\sim10^{-9}$.  To resolve this discrepancy, Hopfield \cite{Hopfield1974} and Ninio \cite{Ninio1975} proposed a network which uses non-equilibrium drive to enhance discrimination by amplifying equilibrium energy differences.  Such networks have been analyzed extensively, extended \cite{Ehrenberg1980,Murugan2012} and generalized \cite{Murugan2014}.

The Hopfield-Ninio scheme achieves its minimum error rate in the slow, quasi-adiabatic regime.  Some years later, Bennett introduced a discrimination scheme which achieves its error minimum in the fast, high-dissipation limit \cite{Bennett1979,Bennett1982}.  An essential difference between these schemes was only recently noted: quasi-adiabatic discrimination amplifies {\it binding energy} differences, whereas high-dissipation discrimination amplifies {\it activation energy} differences.  The two regimes were termed {\it energetic} and {\it kinetic} discrimination (respectively), and argued to be alternate in any single reaction step \cite{Sartori2013}.

We study the general requirements for non-equilibrium discrimination from a geometric perspective.  To this end, we introduce a measure on non-equilibrium systems, called {\it orthogonality}. The measure tightly bounds the degree to which a network's discrimination can be accurately represented by the transition rates near (0-2 links away from) the discriminatory nodes.  We find that local (high orthogonality) discrimination relies on the existence of many effective pathways directed towards the discriminatory nodes.  In contrast, discrimination schemes that depend on global network parameters (low orthogonality) are characterized by having a single dominant path towards the discriminatory nodes.

We show that high orthogonality networks are necessary for kinetic discrimination, whereas low orthogonality networks are necessary for energetic discrimination.  Subject to these orthogonality requirements, {\it both} kinetic and energetic discrimination increase with respect to dissipation.  When the orthogonality requirement conflicts with the dissipation requirement - which can happen in either discriminatory regime - the level of discrimination is non-monotonic with respect to dissipation.  This phenomena is responsible for the non-monotonicity observed in the original Hopfield-Ninio scheme.

We further show that orthogonality can be modulated by changing the drive of a single reaction type in an otherwise fixed network.  Whether dissipation increases or decreases orthogonality depends on whether it is used to drive reactions which increase or decrease the number of effective pathways towards the discriminatory nodes.  By shifting orthogonality, nonequilibrium drive can achieve simple, highly tunable selection between products which are favored by different energy types.

Hopfield noted that the reaction topologies necessary for energetic discrimination appear ubiquitously in biology.  Our results demonstrate how these molecular discrimination mechanisms can explore alternative product spaces by spending energy to transition to the kinetic regime, without cost to their original fidelity.  The degree to which a network can be repurposed in this manner is given by the degree to which its orthogonality can be modulated.  We consider this repurposing proposition in the particular case of localized collections of RNA and protein known as {\it granules} \cite{Brangwynne2009}, whose features bear all of the hallmarks of dissipation-driven orthogonality tuning.
