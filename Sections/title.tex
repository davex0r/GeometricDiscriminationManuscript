\title{Geometric Requirements for Nonequilibrium Discrimination}
\author{Gaurav G. Venkataraman}
\email{Correspondence:  gauravvman@gmail.com or dj333@cam.ac.uk}
\affiliation{Wellcome/CRUK Gurdon Institute, University of Cambridge.\\Tennis Court Rd, Cambridge, CB2 1QN, UK.}
\affiliation{Division of Medicine, University College London.  London WC1E 6BT, UK.}

\author{Eric A. Miska}
\affiliation{Wellcome/CRUK Gurdon Institute, University of Cambridge.\\Tennis Court Rd, Cambridge, CB2 1QN, UK.}
\affiliation{Department of Genetics, University of Cambridge.  Downing Street, Cambridge CB2 3EH, UK.}
\affiliation{Wellcome Sanger Institute, Wellcome Genome Campus, Cambridge CB10 1SA, UK.}

\author{David J. Jordan}
\email{Correspondence: gauravvman@gmail.com or dj333@cam.ac.uk}
\affiliation{Wellcome/CRUK Gurdon Institute, University of Cambridge.\\Tennis Court Rd, Cambridge, CB2 1QN, UK.}
\affiliation{Department of Genetics, University of Cambridge.  Downing Street, Cambridge CB2 3EH, UK.}

\begin{abstract}

We study multistep networks whose steady-state occupancies achieve high sensitivity via thermodynamic drive.  This sensitivity allows the ratios of non-equilibrium steady states to depart far from their equilibrium limit, known as discrimination.  Discrimination is crucial for high fidelity information processing at the molecular scale, where steady-state occupancies correspond to (competing) products of biochemical reactions.  We define an analytically tractable measure on network discrimination schemes, termed orthogonality, which measures the extent to which discrimination is local.  The central proposition of our paper is that discrimination is fundamentally constrained by orthogonality.  We demonstrate that discrimination which amplifies binding energy differences requires low orthogonality, whereas discrimination which amplifies activation energy differences requires high orthogonality.  Subject to orthogonality requirements, {\it both} types of discrimination are maximized by maximizing dissipation.  Dissipation itself can drive orthogonality up or down. When increasing thermodynamic drive conflicts with orthogonality requirements, discrimination is non-monotonic.  We find that, due to its effect on orthogonality, modulating thermodynamic drive alone can sharply select between products which are favored by different energy types, without network fine-tuning.  Biologically, this corresponds to the ability to select between products by driving a single reaction type.  We consider this possibility in the context of liquid-liquid phase separated collections of RNA and protein known as granules, which appear to have precisely the structure required to tune orthogonality via adjusting the rate of ATP hydrolysis.

%We study multistep networks capable of product discrimination inspired by biological \emph{proofreading} mechanisms.   These networks expend free-energy to reach nonequilibrium steady states in which ratios of state occupancies are far from their equilibrium limit.  We introduce a measure on such nonequilibrium systems called orthogonality and show that it provides necessary conditions for proofreading.  In proofreading networks that discriminate with activation energy differences, called kinetic discrimination, this is the limit of high orthogonality. In networks that discriminate with binding energy differences, called energetic discrimination, it is the limit of low orthogonality.  We show that in both kinetic and energetic discrimination, minimum error requires maximizing dissipation in the appropriate orthogonality limits and show how this can lead to non-monotonic behavior in the error-rate as a function of the systems free-energy expenditure.  In addition, we show that in high orthogonality networks there are multiple directed paths to the product state.  In contrast, low orthogonality networks have, in general, a single dominant path to the product.  Finally, we introduce mixed networks, in which one product is favored energetically and the other kinetically.  In such networks, sensitive product switching can be achieved by increasing the free-energy expenditure across a single type of reaction.
\end{abstract}

\maketitle
