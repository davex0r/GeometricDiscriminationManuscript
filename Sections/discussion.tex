We have introduced orthogonality, which measures the degree to which a ratio of non-equilibrium steady states can be represented by rates local to the discriminatory nodes.  We found that orthogonality tends to increase with the number of effectively realizable pathways directed towards the discriminatory nodes.

This connection between orthogonality and realizable pathways underlies its role in non-equilibrium discrimination.  In order to discriminate via binding energies, networks require having a single dominant path along which discrimination occurs via frequently discarding intermediary products.  These processes are inherently processive: discrimination is a global function of discards at sequential steps throughout the graph.  Final product formation is rare, thus slow.  In contrast, discrimination via kinetic barriers is fast.  In the kinetic regime, discrimination relies on creating final products quickly, enabled by distributive networks which have many paths towards the final products.  We thus find that orthogonal networks are necessary for kinetic discrimination, whereas non-orthogonal networks are necessary for energetic discrimination.

It is interesting to consider this result in the context of protein complex assembly~\cite{Murugan2014a}.  Sartori and Leibler~\cite{Sartori2019} have recently proposed that a significant proportion of the discrimination necessary for accurate protein complex assembly can be achieved by equilibrium energy differences in protein-protein interactions.  Our results predict that non-equilibrium mechanisms which amplify these energetic differences should result in complexes being assembled sequentially, and slowly.  If non-equilibrium mechanisms instead amplify kinetic differences to achieve accurate assembly, we expect a complex's component subunits to assemble in many different orders, quickly.

Our results clarify the role of thermodynamic drive in nonequilibrium discrimination.  We find that both kinetic and energetic discrimination are enhanced by increasing dissipation, but are subject to necessary requirements on orthogonality, which itself can be modulated upwards or downwards by free energy expenditure.

By modulating orthogonality with energy expenditure, discriminatory networks can achieve sensitive product switching.  In particular, driving a {\it single} reaction type is sufficient for sharp selection between products, if the products are favored by different energy types and driving shifts the orthogonality of the network.

Biologically, this possibility may be realized in cytoplasmic ribonucleoprotein (RNP) granules~\cite{Brangwynne2009}.  These granules are composed of RNAs and proteins coloclazied in liquid-liquid phase separated droplets.  Their liquid-liquid like components interact promiscuously, and are known to be enriched for multivalent components~\cite{Banani2017}.  RNA contributes to promiscuous granule interactions via both RNA-RNA interactions and serving as a protein scaffold.  Both RNA-RNA interactions and the number of RNA-protein contacts are dependent on RNA secondary structure~\cite{Groot2019}.  It thus appears that RNA secondary structure can modulate the orthogonality of the granule interaction network.

RNA structure is appealing as a modulator of orthogonality because it can be modified by driving a single reaction type.  It has been recently reported that ATP within granules is hydrolyzed by DEAD-box proteins, which remodel RNA by unwinding duplexes \cite{Hondele2019}.  The (ATP-driven) DEAD-box unwinding of RNA has been reported responsible for the dynamic makeup of RNA inside of granules, and for granule dissolution.  It is possible that driving this reaction type can tune the orthogonality of granule interaction networks.

Whether, and in which direction, ATP-driven RNA unwinding tunes orthogonality will depend on the molecular components of the granule.  These components are not fixed; granules constantly exchange material with the local environment and are capable of exchanging components with each other.  This combination of dynamic components and orthogonality driven selection may allow the cell to use existing components to explore new areas of biochemical reaction space.  Such an ability is consistent with the apparent importance of granules in a wide variety of cellular responses to environmental cues, including stress response~\cite{Buchan2009}, transcriptional regulation~\cite{Anderson2009}, and local, activity dependent translation of mRNA at neuronal synapses~\cite{McCann2011, Barbee2006}.