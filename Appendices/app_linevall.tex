\section{Orthogonality of the line versus all-to-all graph}
\label{app:line_v_all}
In this Appendix we demonstrate that the orthogonality of an $N$ node line graph is strictly less than an $N$ node all-to-all connected graph, in the toy case where all rate constants are the same.  The result follows from directly calculating the orthogonality for each topology, which we do in turn.

\begin{prop}[$\Theta$ for a line graph]{For a a line graph with bidrectional connections of equal weight (set to 1 without loss of generality), the orthogonality is given by: $\Theta  = 1 - \sqrt{(N-1)\frac{8}{9}+\frac{1}{36}(N-4)}$. 
}
\end{prop}
\begin{proof}
The result follows from direct computation of $\la i, \ j\ra, \ \forall \  i\neq 1,N.$
There are only two types of nonzero $\la i,j \ra.$  The first type is $\la i, i+1 \ra;$ there exist $2(N-1)$ terms of this type.  The second type is $\la i,i+2 \ra;$ there exist $N-4$ entries of this type.  The first type of nonzero term represents `neighbors.'  The second represents nodes separated by one node, which point at a mutual node.  The two types of inner product have (squared, normalized) values:
\[
\la i, i+1 \ra^2 = \frac{(-\alpha\cdot2\alpha-\alpha\cdot 2\alpha)^2}{(2\alpha^2+4\alpha^2)^2} = \frac{4}{9}
\]
and
\[
\la i, i+2 \ra^2 = \frac{(\alpha^2)^2}{(6\alpha^2)^2} = \frac{1}{36}.
\]
The result follows.
\end{proof}

The all-to-all calculation is slightly more complicated.

\begin{prop}[$\Theta$ for an all-to-all graph]{For an all-to-all connected graph with bidrectional connections of equal weight (set to 1 without loss of generality), the orthogonality is given by: $\Theta  = 1 - \sqrt{ \frac{(N-2)(N-3)}{(N-1)^2}}$. 
}
\end{prop}
\begin{proof}
Let $S$ be the $n$ by $n-2$ matrix formed by removing two of the columns of the Laplacian for this graph.

Because the diagonal elements $(S^TS)_{ii} = 1,$ we need only compute the off-diagonal elements of $S^TS.$  A generic such element resulting from taking the (not normalized) inner product of columns $j, k$ is given by
\[
\begin{aligned}
\langle j, k \rangle &= \sum_{\tristack{i}{i \neq j}{i\neq k}}\theta_{ij}\theta_{ik} - \theta_{jk}\cdot\sum_{\sumstack{i}{i\neq j}} \theta_{ij} - \theta_{kj}\sum_{\sumstack{i}{i\neq k}} \theta_{ik}\\
& = (N-2)\alpha^2-\alpha^2(N-1)-\alpha^2(N-1)\\
&=-\alpha^2N.\\
\end{aligned}
\]
where the first line is a generic expression for the inner product of columns corresponding to connected nodes for matrix elements $\theta_{ij}$ of $S$, and the resulting lines follow from bidirectional all-to-all connectivity with equal rate constants.

We now need to compute the normalization factor:
\[
\begin{aligned}
\left(\norm{j}\norm{k}\right)^2 =& 
\left(\sum_{\eyej}\theta^2_{ij} + \left(\sum_{\eyej}\theta_{ij}\right)^2\right)\\
&\cdot \left(\sum_{\eyek}\theta^2_{ik} + \left(\sum_{\eyek}\theta_{ik}\right)^2\right)\\
&= \left(\alpha^2(N-1) + (N-1)^2\alpha^2\right)^2\\
&= \left(\alpha^2(N^2-N)\right)^2\\
& = \alpha^4(N^2-N)^2
\end{aligned}
\]
where again we have begun with generic terms for the normalization of the inner product of columns of the Laplacian matrix, with  $\theta_{ij}$ representing the elements of $S$.

Putting these together yields the expression for a generic element of $S^TS$:
\[
\begin{aligned}
\frac{\langle j, k \rangle^2}{\left(\norm{i}\norm{j}\right)^2} &= \frac{\alpha^4N^2}{\alpha^4(N^2-N)^2}\\
& = \frac{1}{(N-1)^2}.
\end{aligned}
\]
How many such elements exist?  We know that $S^TS$ is a square $n-2$ length matrix, and we know that the diagonal terms are zero.  We therefore have $(n-2)(n-3)$ entries each equal to $\frac{1}{(N-1)^2}.$  The result follows.
\end{proof}

From the two propositions we can calculate that
\[
\begin{aligned}
\Theta_{\text{all-to-all}} - \Theta_{\text{line}} =&
-\sqrt{ \frac{(N-2)(N-3)}{(N-1)^2}} \\
&+\sqrt{(N-1)\frac{8}{9}+\frac{1}{36}(N-4)}
\end{aligned}
\]

The former (negative) term quickly asymtotes to 1, whereas the latter (positive) term grows as $O(\sqrt{N}).$  We conclude that the orthogonality of the all-to-all graph is greater than the line graph, and this difference is increasing for increasing $N$.

