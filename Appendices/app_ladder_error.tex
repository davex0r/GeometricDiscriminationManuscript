\section{Expressions for Error Rate in the Ladder Graph}
\label{app:ladder_error}
We wish to derive expressions for the error rate of the ladder discrimination scheme in the kinetic and energetic regimes.

A single side of the ladder has structure:
\begin{center}
\schemestart
0 \arrow(0--$y_{s0}$){<=>[$\mathrm{k_{on}}$][$\mathrm{k_{off}}$]}[,,,red] $y_{s0}$
\arrow{<=>[$u$][$d$]}[90,,,red] 
$x_{s0}$ \arrow($x_{s0}$--$x_{s1}$){->[$f$]}[,,,red]$x_{s1}$
\arrow(@$y_{s0}$--$y_{s1}$){<-[][$b$]}[,,,red] $y_{s1}$ \arrow{<=>[$u$][$d$]}[90,,,red]
\arrow(@$y_{s1}$--$y_{s2}$){<-[][$b$]}[,,,red] $y_{s2}$
\arrow{<=>[$u$][$d$]}[90,,,red] 
\arrow(@$x_{s1}$--$x_{s2}$){->[$f$]}[,,,red] $x_{s2}$
\schemestop
\end{center}
where we have dropped the superscripts $d^S, \ u^S$ for clarity.

We will use the Matrix-Tree theorem (MTT), which provides an expression for steady states in terms of {\it spanning trees}~\cite{Wong2018}.  Recall that a {\it spanning tree} of a graph $G$ is is a subgraph which includes every vertex of $G$ and has no cycles (when edge directions ignored).  A spanning tree is said to be {\it rooted} at node $i$ if node $i$ is the only vertex of the subgraph without any outgoing edges.

The MTT provides an expression for the steady state of node $i$ in terms of the sum of the product of the rates of each spanning tree rooted at $i$.  That is:
\begin{equation}\label{eq:mtt}
\rho_i = \sum_{T\in S_i(G) }\left(\prod_{j\stackrel{a}{\to} k\in T}a \right),
\end{equation}
where $S_i(G)$ is the set of all spanning trees of graph $G$ rooted at $i$.

We will exploit the structure of our ladder network in order to simplify this expression.  Our ladder consists of two subgraphs joined at a single node, 0: $G = G_W \oplus_0 G_R$ ($G_R, \ G_W,$ corresponding to subgraphs for the right, wrong products, respectively).  Equation \ref{eq:mtt} implies that
\[
\rho_i(G_W \oplus_0 G_R) =
 \begin{cases}
       \rho_i(G_W)\rho_0(G_R), \ \text{if } i\in G_W\\
      \rho_i(G_R)\rho_0(G_W), \ \text{if } i\in G_R.\\
\end{cases}
\]
%Because the subgraphs share a single node, the kernel element corresponding to node $i$ in subgraph $R$ is given by $\rho_i = \rho(G_R)\rho_0(G_W)$~\cite{Wong2018}.
This gives for the error
\begin{equation}\label{jer_span}
\xi = \frac{\rho_W}{\rho_R} = \frac{\rho_W(G_W)\rho_0(G_R)}{\rho_R(G_R)\rho_0(G_W)},
\end{equation}
where $\rho_W, \rho_R$ represent the kernel elements corresponding to the discriminatory nodes in the upper corner of the wrong, right subgraphs (respectively).

We therefore need only determine analytical expressions for the sums of (products of rate constants of) spanning trees rooted at the top corner and 0 nodes. 
Let's count the trees rooted at $\rho_0(G_S)$ first.  In order for the tree to be rooted at $0$, there are a number of essential arrows:

\bigskip

\begin{center}
\schemestart
0 \arrow(0--$y_{s0}$){<-[][$\mathrm{k_{off}}$]}[,,,red] $y_{s0}$
\arrow{<=>[$u$][$d$]}[90,,,white] 
$x_{s0}$ \arrow($x_{s0}$--$x_{s1}$){->[$f$]}[,,,white]$x_{s1}$
\arrow(@$y_{s0}$--$y_{s1}$){<-[][$b$]}[,,,red] $y_{s1}$ \arrow{<=>[$u$][$d$]}[90,,,white]
\arrow(@$y_{s1}$--$y_{s2}$){<-[][$b$]}[,,,red] $y_{s2}$
\arrow{<-[][$d$]}[90,,,red] 
\arrow(@$x_{s1}$--$x_{s2}$){->[$f$]}[,,,white] $x_{s2}$
\schemestop
\end{center}
without any of which it is impossible to produce a spanning tree rooted at $0.$  The necessity of these arrows comes from the unidirectionality of the $f, b.$

What other arrows are necessary for a spanning tree?  Consider the diagram

\begin{center}
\schemestart
0 \arrow(0--$y_{s0}$){<-[][$\mathrm{k_{off}}$]}[,,,red] $y_{s0}$
\arrow{<-[][$d$]}[90,,dashed,green] 
$x_{s0}$ \arrow($x_{s0}$--$x_{s1}$){->[$f$]}[,,,green]$x_{s1}$
\arrow(@$y_{s0}$--$y_{s1}$){<-[][$b$]}[,,,red] $y_{s1}$ \arrow{<-[][$d$]}[90,,dashed,blue]
\arrow(@$y_{s1}$--$y_{s2}$){<-[][$b$]}[,,,red] $y_{s2}$
\arrow{<-[][$d$]}[90,,,red] 
\arrow(@$x_{s1}$--$x_{s2}$){->[$f$]}[,,,blue] $x_{s2}$
\schemestop
\end{center}

It is necessary and sufficient for a spanning tree rooted at $0$ to contain all of the red arrows, and exactly one of the green arrows and one of the blue arrows.  This holds in general; each loop in a ladder must contribute either a factor of $f$ or $d$ to a spanning tree rooted at 0.

We can thus compute:
\[
\begin{aligned}
\rho_0(G_S) &= k_{\rm off}b^\alpha d \sum_{k=0}^\alpha {\alpha\choose k}f^{\alpha-k}d^k\\
&=k_{\rm off}b^\alpha d (f + d)^\alpha
\end{aligned}
\]
where the second line follows from the Binomial theorem, and where we have set the number of square loops in the ladder portion of the graph to be $\alpha.$

We can now repeat this procedure with spanning trees rooted in the upper corner, with red, blue, and green as before:

\begin{center}
\schemestart
0 \arrow(0--$y_{s0}$){->[$\mathrm{k_{on}}$][]}[,,,red] $y_{s0}$
\arrow{->[$u$][$$]}[90,,,red] 
$x_{s0}$ \arrow($x_{s0}$--$x_{s1}$){->[$f$]}[,,,red]$x_{s1}$
\arrow(@$y_{s0}$--$y_{s1}$){<-[][$b$]}[,,dashed,green] $y_{s1}$ \arrow{->[$u$][$$]}[90,,dashed,green]
\arrow(@$y_{s1}$--$y_{s2}$){<-[][$b$]}[,,dashed,blue] $y_{s2}$
\arrow{->[$u$][$$]}[90,,dashed,blue] 
\arrow(@$x_{s1}$--$x_{s2}$){->[$f$]}[,,,red] $x_{s2}$
\schemestop
\end{center}


Which gives us:
\[
\begin{aligned}
\rho_S(G_S) &= k_{\rm on}f^\alpha u \sum_{k=0}^\alpha {\alpha\choose k}b^{\alpha-k}u^k\\
&=k_{\rm on}f^\alpha u (b + u)^\alpha.
\end{aligned}
\]
Note that in comparison to the last expression, we have merely made the substitutions: $b\to f, \ f \to b, \ d \to u, \  u \to d.$  Plus $k_{\rm off}\to k_{\rm on},$ of course.

Returning to our expression for the error gives
\[
\begin{aligned}
\xi &= \frac{\rho_W}{\rho_R} = \frac{\rho_W(G_W)\rho_0(G_R)}{\rho_R(G_R)\rho_0(G_W)}\\
&= \frac{k_{\rm on}k_{\rm off}\  f_W^\alpha \ u_W \ b_R^\alpha \  d_R \  (u_W + b_W)^\alpha(f_R+d_R)^\alpha }{k_{\rm on}k_{\rm off}\  f_R^\alpha \ u_R \ b_W^\alpha \  d_W \  (u_R + b_R)^\alpha(f_W+d_W)^\alpha }
\end{aligned}
\]
where we have denoted variables coming from the `right' and `wrong' sides of the ladder with subscripts $R$ and $W,$ respectively.  We can do some cancellation (b = $b_R$ = $b_W$ and $f = f_R = f_W$) to arrive at:
\[
\xi = \frac{d_Ru_W(u_W+b)^{\alpha}(f+d_R)^\alpha}{d_Wu_R(u_R+b)^{\alpha}(f+d_W)^\alpha}.
\]
In the energetic regime we have that $u_R = u_W,$ and that $d_W = d_Re^\gamma:$
\[
\xi_{\rm energetic} = \frac{(f+d_R)^\alpha}{e^\gamma(f+d_Re^\gamma)^\alpha}.
\]
%
In the kinetic regime, we have that $d_R = d_We^\delta, \ u_R = u_We^\delta,$ giving 
%
\[
\xi_{\rm kinetic} =  \frac{(u+b)^\alpha(f+de^\delta)^\alpha}{(ue^\delta+b)^\alpha(f+d)^\alpha}.
\]
