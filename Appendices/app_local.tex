\section{Orthogonality implies locality}
\label{app_local}
In this Appendix we wish to demonstrate that orthogonal networks have discrimination indices that do not depend on rates distant from the nodes being discriminated between.

For an orthogonal network with Laplacian $\mL$ and discrimination nodes $i, j$, we have that
\[
\frac{\rho_i}{\rho_j} \approx \frac{\norm{v_j - \proj_\sij^{\text{orth}}(v_j)}}{\norm{v_i - \proj_\sij^{\text{orth}}(v_i)}}
\]
where $v_x$ is the column of $\mL$ which corresponds to node $x$, and where
\[
\proj_\sij^{\text{orth}}(v_i) = \sum_{l\in \sij}\langle v_i, v_l \rangle v_l
\]
Note that $\langle v_i, v_l \rangle\neq 0$ only if the $i$th and $l$th nodes on the digraph associated with $\mL$ are directly linked or direct arrows at at least one mutual node $k$.  The ratio ${\rho_i}/{\rho_j}$ is thus only a function of the rates associated with nodes at most two links away from states $i, j,$ along with rates associated with states $i,j$ themselves. 