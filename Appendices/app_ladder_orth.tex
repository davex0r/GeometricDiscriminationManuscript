\section{Orthogonality and Error in the Ladder Graph}
\label{app:ladder_orth}

We first derive the discriminatory limit in the energetic regime.  Recall that

\subsection{Energetic regime}
\[
\xi_{\rm energetic} = \frac{(f+d_R)^\alpha}{e^\gamma(f+d_Re^\gamma)^\alpha}.
\]
The substitution $\eta=\frac{d_R}{f}$ gives
\[
\xi_{\rm energetic} = \frac{(1+\eta)^\alpha}{e^\gamma(1+\eta e^\gamma)^\alpha}
\]
from which read off that proofreading requires $\eta$ to be large.  This corresponds to the intuition that the rate of discards must be large with respect the reaction speed.  

We must now demonstrate that orthogonality is decreasing as $\eta$ becomes large.

%With this in mind, we would like to calculate the orthogonality and show that in the energetic regime, it is a decreasing function of $\eta$

As in the Ninio-Hopfield case, we will use the notation $\sum{s^2_{i,j}}$ to denote the squared, normalized inner product between columns $i, j$ in Matrix $\mathcal{L}^{a,b}$ formed by deleting the columns corresponding to the discriminatory nodes $a, b$ from the full Laplacian for this graph.

For any given loop of the ladder, these terms are given by
\[
\begin{aligned}
 \la x_{si}, y_{s(i+1)} \ra^2 &=\frac{(b d+f u)^2}{4 \left(b^2+b u+u^2\right) \left(d^2+d f+f^2\right)}\\
 \la y_{si}, y_{s(i+1)} \ra^2 &= \frac{b^2 (b+u)^2}{4 \left(b^2 + bu+ u^2\right)^2}\\
 \la x_{si}, x_{s(i+1)}\ra^2 &=\frac{f^2 (d+f)^2}{4 \left(d^2+d f+f^2\right)^2}\\
 \la x_{si}, y_{si}\ra^2 &=\frac{(b d+2 d u+f u)^2}{4 \left(b^2+b u+u^2\right) \left(d^2+d f+f^2\right)}.
 \end{aligned}
\]
For $N$ loops, there will be $N$ of each of these terms except for $ \la x_{si}, x_{s(i+1)}\ra^2$ for which there will be $(N-1)$ for each side of the ladder.  In addition, there will be two terms that originate from the reactant node (note in this case we are considering a slightly altered graph, where $k_{off}=b$ and $k_{on}=f$, and $k_{on}$ connects 0 to $x_{s0}$).   These are given as
\[
\begin{aligned}
\la 0, x_{s0} \ra^2 &= \frac{(2 b-u)^2}{12 \left(b^2+b u+u^2\right)} \\
\la 0, y_{s0} \ra^2 &= \frac{(d+f)^2}{6 \left(d^2+(d+f)^2+f^2\right)}.\\
\end{aligned}
\]
Recall that in the energetic regime, our effective parameter of interest if $\eta=d/f$, noting that  $\la y_{si}, y_{s(i+1)}$ and $\la 0, x_{s0} \ra^2$ are not functions of $\eta$ and making this substitution along with the another substitute $\phi=u/b$ gives %the following expressions,
\[
\begin{aligned}
 \la x_{si}, y_{s(i+1)} \ra^2 &=\frac{(\eta+\phi)^2}{4 \left(1+ \phi+\phi^2\right) \left(1+\eta+\eta^2\right)}\\
 \la x_{si}, x_{s(i+1)}\ra^2 &=\frac{(\eta+1)^2}{4 \left(1+\eta+\eta^2\right)^2}\\
 \la x_{si}, y_{si}\ra^2 &=\frac{(\eta+2 \eta \phi+ \phi)^2}{4 \left(1+\phi+\phi^2\right) \left(1+\eta+\eta^2\right)}\\
 \la 0, y_{s0} \ra^2 &=\frac{(\eta+1)^2}{12 \left(1+\eta+\eta^2\right)}\\
 \end{aligned}
\]
We will set $\phi\to0$ for convenience.  In this limit we have:
%What values should $\phi$ take for efficient proofreading in the energetic regime?  $u$ is the rate of rescues this rate must be small relative to the rate of backtracking for efficient proofreading which implies that $\phi\to 0$.  In this limit, the expressions become.
\[
\begin{aligned}
\la x_{si}, y_{s(i+1)} \ra^2 &=\frac{\eta^2}{4 \left(1+\eta+\eta^2\right)}\\
\la x_{si}, x_{s(i+1)}\ra^2 &=\frac{(\eta+1)^2}{4 \left(1+\eta+\eta^2\right)^2}\\
\la x_{si}, y_{si}\ra^2 &=\frac{\eta^2}{4 \left(1+\eta+\eta^2\right)}\\
\la 0, y_{s0} \ra^2 &=\frac{(\eta+1)^2}{12 \left(1+\eta+\eta^2\right)}\\
\end{aligned}
\]
which take values 0, 1/4, 0, and 1/12 in the limit $\eta\to0$ and 1/4, 0, 1/4, and 1/12 in the limit $\eta\to\infty$ For $N$ loops, we will have N terms of the first and third type, and $N-1$ terms of the second type.  The last term is unchanged in these limits.  This gives the desired result, 
\[
\lim_{\eta\to 0}\sum s^2_{i,j} \propto \frac{N-1}{4}<\lim_{\eta\to\infty}\sum s^2_{i,j} \propto \frac{2N}{4}.
\]
Finally, we consider the case when $\phi\to\infty$.  Note that $\la x_{si}, x_{s(i+1)}\ra^2$ terms are not functions of $\phi$.  The two remaining terms to consider are,
\[
\begin{aligned}
 \la x_{si}, y_{s(i+1)} \ra^2 &=\frac{(\eta+\phi)^2}{4 \left(1+ \phi+\phi^2\right) \left(1+\eta+\eta^2\right)}\\
 \la x_{si}, y_{si}\ra^2 &=\frac{(\eta+2 \eta \phi+ \phi)^2}{4 \left(1+\phi+\phi^2\right) \left(1+\eta+\eta^2\right)}\\
 \end{aligned}
\]
which in the $\phi\to\infty$ limit become,
\[
\begin{aligned}
 \la x_{si}, y_{s(i+1)} \ra^2 &=\frac{1}{4 \eta ^2+4 \eta +4}\\
 \la x_{si}, y_{si}\ra^2 &=\frac{4 \eta ^2+4 \eta +1}{4 \eta ^2+4 \eta +4}\\
 \end{aligned}
\]
Combining these term yields,
\[
\lim_{\eta\to 0}\sum s^2_{i,j} \propto \frac{2N-1}{4}<\lim_{\eta\to\infty}\sum s^2_{i,j} \propto \frac{4N}{4}.
\]

We can directly compute that $\sum s^2_{i,j} $ is monotonically increasing in both the $\phi\to0$ and $\phi\to\infty$ limits.

%To see that the orthogonality is monotonically decreasing in $\eta$, we need to prove that $\sum s^2{i,j} $ is monotonically increasing.  To do that, we combine all the terms that contribute to the sum with their appropriate multiplicities and take the derivative:
%\[
%\frac{d}{d\eta}\sum s^2{i,j}  = \frac{\eta  (\eta +2) \left(\eta +\eta ^2 N+1\right)}{2 \left(\eta ^2+\eta +1\right)^3}
%\]
%which is strictly positive.  

\subsubsection{$\phi$ does not affect orthogonality in the $f\ll d$ limit}

Before turning to the kinetic regime, we demonstrate that $\phi$ does not affect orthogonality in the energetic discrimination limit.

%In the limit of $f\ll d$, the ratio parameter $\eta=d/f\to\infty$.  In this limit, the orthogonality does not depend on $u$ or $b$, and thus does not depend on the ratio parameter $\phi=u/b$.  

We examine the elements $s^2_{i,j}$ that depend on $\phi$ in the $\eta\to\infty$ discriminatory limit.  Before taking the limit, we have
\[
\begin{aligned}
\la x_{si}, y_{s(i+1)} \ra^2 &=\frac{(\eta+\phi)^2}{4 \left(1+ \phi+\phi^2\right) \left(1+\eta+\eta^2\right)}\\
\la x_{si}, y_{si}\ra^2 &=\frac{(\eta+2 \eta \phi+ \phi)^2}{4 \left(1+\phi+\phi^2\right) \left(1+\eta+\eta^2\right)}\\
\la y_{si}, y_{s(i+1)}\ra^2 &=\frac{(\phi +1)^2}{4 \left(\phi ^2+\phi +1\right)^2}.\\
\end{aligned}
\]
In the $\eta\to\infty$ limit these become
\[
\begin{aligned}
\la x_{si}, y_{s(i+1)} \ra &=\sqrt{\frac{1}{4 \phi ^2+4 \phi +4}}\\
\la x_{si}, y_{si}\ra &=\sqrt{\frac{4 \phi ^2+4 \phi +1}{4 \phi ^2+4 \phi +4}}\\
\la y_{si}, y_{s(i+1)}\ra &=\frac{(\phi +1)}{2 \left(\phi ^2+\phi +1\right)}.\\
\end{aligned}
\]
Now we must evaluate  these in the limits of $\phi\to 0$ and $\phi\to\infty$, the first term goes from 1/2 to 0 as $\phi\to\infty$.  The second term goes from 1/2 to 1 and the third term goes from 1/2 to 0.  Because each loop consists of two of the second type term and one each of the first and third type term, the sum is the same in each limit.  

In the full expression for orthogonality, we do observe a small non-constant dependence on $\phi$, but this is marginal and strictly decreases the orthogonality, thereby reinforcing our notion that $\phi$ cannot be used to increased realizable pathways in the low $f$ regime.  

\subsection{Kinetic regime}

We now need to demonstrate that 

\[
\xi_{\rm kinetic} =  \frac{(u+b)^\alpha(f+de^\delta)^\alpha}{(ue^\delta+b)^\alpha(f+d)^\alpha}.
\]
Define $\eta = d/f, \ \phi = u/b$ as before.
\[
\xi_{\rm kinetic} =  \frac{(\phi+1)^\alpha(1+\eta e^\delta)^\alpha}{(\phi e^\delta+1)^\alpha(\eta+1)^\alpha}.
\]
which attains its minimum of $e^{-\alpha\delta}$ in the limit $\phi \to \infty, \ \eta \to 0.$  
The previous sections demonstrated that orthogonality is increasing in these limits.

%These are exactly opposite the constraints for proofreading in the energetic regime, and we can conclude that because orthogonality will be increasing in these limits.  
